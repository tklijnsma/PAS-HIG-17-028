\section{Introduction}
\label{sec:introduction}

The Higgs boson, whose existence is predicted by the Brout-Englert-Higgs mechanism~\cite{Higgs:1964pj,Englert:1964et,Guralnik:1964eu}, is responsible for electroweak symmetry breaking in the standard model (SM).
% 
Since its discovery~\cite{Aad:2012tfa,Chatrchyan:2012xdj} at the CERN LHC, extensive effort has been dedicated to the measurement of its properties and couplings.

% Measurements have been made of both differential and inclusive cross sections for Higgs boson production but most of the information about its properties has come from measurements of total yields~\cite{Khachatryan:2016vau,Aad:2015gba,Khachatryan:2014jba,ATLAS:2017ovn,CMS:2018lkl}, which have thus far been found to be consistent with SM predictions.
% % 
% In this analysis we study the differential cross sections for the Higgs boson, which enables measurements of deviations in differential distributions that are not accessible via an inclusive measurement.
% % 
% By varying the Higgs couplings, the strength with which quarks and other bosons couple to the Higgs boson, significant distortions in the shapes of differential cross sections appear, in particular for the transverse momentum distribution.
% % 
In this analysis we study the differential cross sections for the production of Higgs bosons. 
% 
Significant distortions of the shapes of differential cross sections appear when the Higgs couplings to quarks and to other bosons are varied.
% 
These distortions are particularly pronounced in the transverse momentum distribution.
% 
Measurements of differential distributions therefore provide information on the couplings that cannot be obtained from inclusive measurements~\cite{Khachatryan:2016vau,Aad:2015gba,Khachatryan:2014jba,ATLAS:2017ovn,CMS:2018lkl}.
% 
By fitting parametrized predicted spectra to a state-of-the-art combination of differential cross sections, one can constrain Higgs couplings.


While the couplings to the top ($y_t$) and bottom ($y_b$) quarks are known with fair precision, there is still a relatively large uncertainty on the couplings to lighter quarks such as the charm.
% 
Increasing the precision on the Higgs boson couplings can provide fundamental insight into the SM, as they are free parameters in the SM Lagrangian and are sensitive to a range of beyond the SM theories
~\cite{Dimopoulos:1981zb,Witten:1981nf}.
% 
A proof-of-concept study determining limits on the modification of the Higgs coupling to the charm quark $\kappa_c$ from the Higgs boson transverse momentum distribution was performed in Ref.~\cite{Bishara:2016jga}
% using ATLAS data from Run I,
using $20.3$\ifb of data collected by ATLAS at a centre-of-mass energy $\sqrt{s}=8$~\cite{Aad:2015lha},
yielding the overall bounds $\kappa_c \in [ -16, 18 ]$.
% 
Direct detection of $H \rightarrow J/\psi$, which relies on charm-tagging, leads to significantly weaker limits: A recent study from the ATLAS Collaboration~\cite{Aaboud:2018fhh} quotes an expected upper limit on $\sigma(pp\rightarrow ZH) \times \text{BR}(H\rightarrow cc)$ of $150^{+80}_{-40}$ times the SM value at 95\% CL, which corresponds to a limit about 4 times weaker.
% 
An analysis by the ATLAS Collaboration of $H \rightarrow J/\psi\gamma$~\cite{Aad:2015sda}, involving the detection of an associated photon, yields $\left|\kappa_c\right|<429$~\cite{Koenig:2015pha} at 95\% CL.
% 
Limits from detection of $pp \rightarrow VH(\rightarrow cc)$ at LHCb~\cite{LHCb:2016yxg} and $pp \rightarrow Hc$~\cite{Brivio:2015fxa} do not yield competitive limits at the current integrated luminosity, although projections to higher luminosities indicate they may do so in the future.
% 
% \arc{What are the physics model that could produce anomalous kc, kb, kt, kg ? It could be interesting to include some examples in the introduction.}
% 
% \tk{Postpone this comment}


Both the ATLAS and CMS Collaborations have reported measurements of differential cross sections at $\sqrt{s}=8$ and $13$\TeV~\cite{Aad:2014lwa,Khachatryan:2015rxa,Aad:2014tca,Khachatryan:2015yvw,Aad:2016lvc,Khachatryan:2016vnn,Aaboud:2017oem,CMS_AN_2016-442,ATLAS:2018uoi,Aaboud:2018xdt}.
% 
We report the measurements of differential cross sections obtained from a combination of results from the CMS Collaboration in the $\hgg$~\cite{CMS_AN_2017-299}, $\hzztofourl$~\cite{CMS_AN_2016-442} and $\hbb$~\cite{CMS_AN_2016-366} decay channels.
% 
The results of the individual decay channels were all obtained at $\sqrt{s}=13\,$TeV, using a data set corresponding to an integrated luminosity of about $35.9\,\text{fb}^{-1}$.
% 
The differential cross sections for the following observables are combined: the Higgs boson transverse momentum $\pth$, the number of reconstructed hadronic jets $\njets$, the Higgs boson rapidity $\absy$, and the transverse momentum of the leading jet $\ptjet$.
% 
Furthermore, we interpret the Higgs boson transverse momentum spectrum in terms of Higgs boson couplings.
% 
In order to take into account the many degrees of freedom offered by the Higgs coupling modifier framework~\cite{LHCHiggsCrossSectionWorkingGroup:2012nn}, multiple couplings are varied simultaneously.
% 
We present results obtained by varying simultaneously
% 
(i) the modifier of the Higgs coupling to the charm quark $\kappa_c$ and of the bottom quark $\kappa_b$,
% 
(ii) the modifier of the Higgs coupling to the top quark $\kappa_t$ and the coefficient $\cg$ of the anomalous direct coupling to the gluon in the heavy top mass limit,
% 
and (iii) $\kappa_t$ and $\kappa_b$.


